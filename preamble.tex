% My stuff

\usepackage[]{algorithm2e}
%\usepackage{algorithmic}
%\usepackage{algpseudocode}
\usepackage[active]{srcltx}
%\usepackage[scale=0.8]{geometry}
\usepackage{amsmath}
\usepackage{amssymb}
\usepackage{amsfonts}
\usepackage{amsthm}
\usepackage{bm}
%\usepackage{bbding}
%\theoremstyle{plain}

\usepackage{microtype}
\usepackage{graphicx}
\usepackage{float}
\usepackage[T1]{fontenc}
\usepackage[utf8]{inputenc}
\usepackage{authblk}
%\usepackage{a4wide}
\usepackage{caption}
\usepackage{subcaption}
%\usepackage{times}
%\usepackage[table]{xcolor}
%\usepackage{xspace}
\usepackage{tikz}
\usetikzlibrary{plotmarks}
\usetikzlibrary{decorations.markings}
\usepackage{chngpage}
%\usepackage[latin1]{inputenc}
\usepackage[american]{babel}
\usepackage{graphicx}
%\usepackage{epsfig}
\usepackage{color}
\usepackage{multirow}
\hyphenation{mono-ton-icity}
\usepackage[margin=0in,font=small,labelfont=bf,indention=0cm]{caption}  % for hanging captions
\usepackage{latexsym}
%\usepackage{pstricks}
%\usepackage{fancyhdr}
\usepackage{setspace}   % just for \NOTE
\usepackage{framed}

%\newcommand{\NOTE}[2]{\marginpar{\setstretch{0.43}\textcolor{blue}{\bf\tiny #1: }\textcolor{red}{\bf\tiny #2}}}
\newcommand{\NOTE}[2]{$^{\textcolor{red}\clubsuit}$\marginpar{\setstretch{0.43}\textcolor{blue}{\bf\tiny #1: }\textcolor{red}{\bf\tiny #2}}}
\newcommand{\red}[1]{\textcolor{red}{#1}}

\newcommand{\NOTEB}[2]{$^{\textcolor{blue}\clubsuit}$\marginpar{\setstretch{0.43}\textcolor{blue}{\bf\tiny #1: }\textcolor{blue}{\bf\tiny #2}}}

% cancelling \NOTE, \NOTEB, and \red
%\renewcommand{\NOTE}[2]{}
%\renewcommand{\red}[1]{#1}
\renewcommand{\NOTEB}[2]{}

\newcommand{\Bin}{\mathsf{Bin}}
\newcommand{\Exp}{\mathsf{Exp}}
\newcommand{\Geo}{\mathsf{Geo}}
\newcommand{\Uni}{\mathsf{Uni}}
\newcommand{\Ber}{\mathsf{Ber}}
\newcommand{\Poi}{\mathsf{Poi}}
\renewcommand{\phi}{r}

\newcommand{\ARR}{\mathsf{ARR}}
\newcommand{\im}{\mathsf{i}}
\usepackage[numbers,sort&compress,longnamesfirst,sectionbib]{natbib}
\bibliographystyle{abbrv}

\floatstyle{ruled}
\newfloat{algorithm}{htbp}{loa}
\floatname{algorithm}{Algorithm}


%\renewcommand{\labelitemi}{\textbullet}
%\renewcommand{\labelenumi}{(\arabic{enumi})}  % or: {(\roman{enumi})}

\renewcommand{\Xi}{\xi}

\newtheorem{thm}{Theorem}  %[chapter]
%\newtheorem{fact}[thm]{Fact}
\newtheorem{lem}[thm]{Lemma}
\newtheorem{obs}[thm]{Observation}
\newtheorem{clm}[thm]{Claim}
\newtheorem{cor}[thm]{Corollary}
\newtheorem{fac}[thm]{Fact}
\newtheorem{rem}[thm]{Remark}
\newtheorem{ass}[thm]{Assumption}
\newtheorem{pro}[thm]{Proposition}
\newtheorem{con}[thm]{Conjecture}
\newtheorem{exm}[thm]{Example}
%\theoremstyle{defn}
%\newtheorem{pbl}{Open Problem}
%\newtheorem{definition}{Definition}
\newtheorem{remark}{Remark}
%\newtheorem{lemma}{Lemma}
%\newtheorem{theorem}{Theorem}
\newtheorem{observation}{Observation}
\newtheorem{claim}{Claim}
%\newtheorem{problem}{Problem}

\numberwithin{thm}{section}
%\numberwithin{defi}{section}
\numberwithin{equation}{section}

\newcommand{\tsum}{\textstyle\sum}

\newcommand{\shortthmref}[1]{Thm.~\ref{thm:#1}}
\newcommand{\thmref}[1]{Theorem~\ref{thm:#1}}
\newcommand{\thmrefs}[2]{Theorems~\ref{thm:#1} and~\ref{thm:#2}}
\newcommand{\thmrefss}[3]{Theorems~\ref{thm:#1},~\ref{thm:#2} and~\ref{thm:#3}}
\newcommand{\proref}[1]{Proposition~\ref{pro:#1}}
\newcommand{\prorefs}[2]{Propositions~\ref{pro:#1} and~\ref{pro:#2}}
\newcommand{\lemref}[1]{Lemma~\ref{lem:#1}}
\newcommand{\lemrefs}[2]{Lemmas~\ref{lem:#1} and~\ref{lem:#2}}
\newcommand{\lemrefss}[3]{Lemmas~\ref{lem:#1},~\ref{lem:#2}, and~\ref{lem:#3}}
\newcommand{\corref}[1]{Corollary~\ref{cor:#1}}
\newcommand{\obsref}[1]{Observation~\ref{obs:#1}}
\newcommand{\clmref}[1]{Claim~\ref{clm:#1}}
\newcommand{\defref}[1]{Definition~\ref{def:#1}}
\newcommand{\defrefs}[2]{Definitions~\ref{def:#1} and~\ref{def:#2}}
\newcommand{\assref}[1]{Assumption~\eqref{ass:#1}}
\newcommand{\conref}[1]{Conjecture~\ref{con:#1}}
\newcommand{\figref}[1]{Figure~\ref{fig:#1}}
\newcommand{\figrefs}[2]{Figures~\ref{fig:#1} and~\ref{fig:#2}}
\newcommand{\tabref}[1]{Table~\ref{tab:#1}}
\newcommand{\secref}[1]{Section~\ref{sec:#1}}
\newcommand{\secrefs}[2]{Sections~\ref{sec:#1} and~\ref{sec:#2}}
\newcommand{\appref}[1]{Appendix~\ref{app:#1}}
\newcommand{\facref}[1]{Fact~\ref{fac:#1}}
\newcommand{\remref}[1]{Remark~\ref{rem:#1}}
\newcommand{\charef}[1]{Chapter~\ref{cha:#1}}
\newcommand{\eq}[1]{Equation~\eqref{eq:#1}}
\newcommand{\eqs}[2]{Equations~\eqref{eq:#1} and~\eqref{eq:#2}}
\newcommand{\eqss}[3]{Equations~\eqref{eq:#1},~\eqref{eq:#2}, and~\eqref{eq:#3}}
\newcommand{\Eqs}[2]{Equations~\eqref{eq:#1} -~\eqref{eq:#2}}
\newcommand{\propref}[1]{P\ref{prop:#1}}
\newcommand{\ineq}[1]{Inequality~\eqref{eq:#1}}
\newcommand{\ineqs}[2]{Inequalities~\eqref{eq:#1} and \eqref{eq:#2}}
\newcommand{\exmref}[1]{Example~\ref{exm:#1}}
\newcommand{\constref}[1]{Constraint~\ref{const:#1}}
\newcommand{\constrefs}[2]{Constraints~\ref{const:#1} and~\ref{const:#2}}
\newcommand{\constrefss}[2]{Constraints~\ref{const:#1}--\ref{const:#2}}
\newcommand{\factref}[1]{Fact~\ref{fact:#1}}
\newcommand{\condref}[1]{Condition~\ref{cond:#1}}
\newcommand{\chapref}[1]{Chapter~\ref{chap:#1}}
\newcommand{\chaprefs[2]}{Chapters~\ref{chap:#1} and~\ref{chap:#2}}
\newcommand{\algref}[1]{Algorithm~\ref{alg:#1}}
\newcommand{\tblref}[1]{Table~\ref{tbl:#1}}
\newcommand{\tblrefs}[2]{Tables~\ref{tbl:#1} and ~\ref{tbl:#2}}

\renewcommand{\Pr}[1]{\ensuremath{\operatorname{\mathbf{Pr}}\left[#1\right]}}
\newcommand{\Prbig}[1]{\ensuremath{\operatorname{\mathbf{Pr}}\big[#1\big]}}
\newcommand{\PrBig}[1]{\ensuremath{\operatorname{\mathbf{Pr}}\Big[#1\Big]}}
\newcommand{\Pro}[1]{\ensuremath{\operatorname{\mathbf{Pr}}\left[#1\right]}}
\newcommand{\Ex}[1]{\ensuremath{\operatorname{\mathbf{E}}\left[#1\right]}}
\newcommand{\Exbig}[1]{\ensuremath{\operatorname{\mathbf{E}}\big[#1\big]}}
\newcommand{\ExBig}[1]{\ensuremath{\operatorname{\mathbf{E}}\Big[#1\Big]}}
\newcommand{\Exbigg}[1]{\ensuremath{\operatorname{\mathbf{E}}\bigg[#1\bigg]}}
\newcommand{\ExBigg}[1]{\ensuremath{\operatorname{\mathbf{E}}\Bigg[#1\Bigg]}}
\newcommand{\floor}[1]{\ensuremath{\left \lfloor #1 \right \rfloor}}
\newcommand{\ceil}[1]{\ensuremath{\left \lceil #1 \right \rceil}}

\newcommand{\Znd}{{\mathbb{Z}^{r}_{\neq0}}}
\newcommand{\Nnd}{{\mathbb{N}^{r}_{\neq0}}}
\newcommand{\R}{\mathbb{R}}
\newcommand{\Z}{\mathbb{Z}}
\newcommand{\N}{\mathbb{N}}

\newcommand{\avgXi}{\ensuremath{\overline{\Xi}}}
%\newcommand{\avgX}{\ensuremath{\overline{X}}}
\newcommand{\one}{\textbf{1}}

\newcommand{\bA}{\ensuremath{\mathbb{A}}}
\newcommand{\bB}{\ensuremath{\mathbb{B}}}
\newcommand{\bC}{\ensuremath{\mathbb{C}}}
\newcommand{\bD}{\ensuremath{\mathbb{D}}}
\newcommand{\bE}{\ensuremath{\mathbb{E}}}
\newcommand{\bF}{\ensuremath{\mathbb{F}}}
\newcommand{\bG}{\ensuremath{\mathbb{G}}}
\newcommand{\bH}{\ensuremath{\mathbb{H}}}
\newcommand{\bI}{\ensuremath{\mathbb{I}}}
\newcommand{\bJ}{\ensuremath{\mathbb{J}}}
\newcommand{\bK}{\ensuremath{\mathbb{K}}}
\newcommand{\bL}{\ensuremath{\mathbb{L}}}
\newcommand{\bM}{\ensuremath{\mathbb{M}}}
\newcommand{\bN}{\ensuremath{\mathbb{N}}}
\newcommand{\bO}{\ensuremath{\mathbb{O}}}
\newcommand{\bP}{\ensuremath{\mathbb{P}}}
\newcommand{\bQ}{\ensuremath{\mathbb{Q}}}
\newcommand{\bR}{\ensuremath{\mathbb{R}}}
\newcommand{\bS}{\ensuremath{\mathbb{S}}}
\newcommand{\bT}{\ensuremath{\mathbb{T}}}
\newcommand{\bU}{\ensuremath{\mathbb{U}}}
\newcommand{\bV}{\ensuremath{\mathbb{V}}}
\newcommand{\bW}{\ensuremath{\mathbb{W}}}
\newcommand{\bX}{\ensuremath{\mathbb{X}}}
\newcommand{\bY}{\ensuremath{\mathbb{Y}}}
\newcommand{\bZ}{\ensuremath{\mathbb{Z}}}

\newcommand{\cA}{\ensuremath{\mathcal{A}}}
\newcommand{\cB}{\ensuremath{\mathcal{B}}}
\newcommand{\cC}{\ensuremath{\mathcal{C}}}
\newcommand{\cD}{\ensuremath{\mathcal{D}}}
\newcommand{\cE}{\ensuremath{\mathcal{E}}}
\newcommand{\cF}{\ensuremath{\mathcal{F}}}
\newcommand{\cG}{\ensuremath{\mathcal{G}}}
\newcommand{\cH}{\ensuremath{\mathcal{H}}}
\newcommand{\cI}{\ensuremath{\mathcal{I}}}
\newcommand{\cJ}{\ensuremath{\mathcal{J}}}
\newcommand{\cK}{\ensuremath{\mathcal{K}}}
\newcommand{\cL}{\ensuremath{\mathcal{L}}}
\newcommand{\cM}{\ensuremath{\mathcal{M}}}
\newcommand{\cN}{\ensuremath{\mathcal{N}}}
\newcommand{\cO}{\ensuremath{\mathcal{O}}}
\newcommand{\cP}{\ensuremath{\mathcal{P}}}
\newcommand{\cQ}{\ensuremath{\mathcal{Q}}}
\newcommand{\cR}{\ensuremath{\mathcal{R}}}
\newcommand{\cS}{\ensuremath{\mathcal{S}}}
\newcommand{\cT}{\ensuremath{\mathcal{T}}}
\newcommand{\cU}{\ensuremath{\mathcal{U}}}
\newcommand{\cV}{\ensuremath{\mathcal{V}}}
\newcommand{\cW}{\ensuremath{\mathcal{W}}}
\newcommand{\cX}{\ensuremath{\mathcal{X}}}
\newcommand{\cY}{\ensuremath{\mathcal{Y}}}
\newcommand{\cZ}{\ensuremath{\mathcal{Z}}}

\newcommand{\bcA}{\ensuremath{\boldsymbol{\mathcal{A}}}}
\newcommand{\bcB}{\ensuremath{\boldsymbol{\mathcal{B}}}}
\newcommand{\bcC}{\ensuremath{\boldsymbol{\mathcal{C}}}}
\newcommand{\bcD}{\ensuremath{\boldsymbol{\mathcal{D}}}}
\newcommand{\bcE}{\ensuremath{\boldsymbol{\mathcal{E}}}}
\newcommand{\bcF}{\ensuremath{\boldsymbol{\mathcal{F}}}}
\newcommand{\bcG}{\ensuremath{\boldsymbol{\mathcal{G}}}}
\newcommand{\bcH}{\ensuremath{\boldsymbol{\mathcal{H}}}}
\newcommand{\bcI}{\ensuremath{\boldsymbol{\mathcal{I}}}}
\newcommand{\bcJ}{\ensuremath{\boldsymbol{\mathcal{J}}}}
\newcommand{\bcK}{\ensuremath{\boldsymbol{\mathcal{K}}}}
\newcommand{\bcL}{\ensuremath{\boldsymbol{\mathcal{L}}}}
\newcommand{\bcM}{\ensuremath{\boldsymbol{\mathcal{M}}}}
\newcommand{\bcN}{\ensuremath{\boldsymbol{\mathcal{N}}}}
\newcommand{\bcO}{\ensuremath{\boldsymbol{\mathcal{O}}}}
\newcommand{\bcP}{\ensuremath{\boldsymbol{\mathcal{P}}}}
\newcommand{\bcQ}{\ensuremath{\boldsymbol{\mathcal{Q}}}}
\newcommand{\bcR}{\ensuremath{\boldsymbol{\mathcal{R}}}}
\newcommand{\bcS}{\ensuremath{\boldsymbol{\mathcal{S}}}}
\newcommand{\bcT}{\ensuremath{\boldsymbol{\mathcal{T}}}}
\newcommand{\bcU}{\ensuremath{\boldsymbol{\mathcal{U}}}}
\newcommand{\bcV}{\ensuremath{\boldsymbol{\mathcal{V}}}}
\newcommand{\bcW}{\ensuremath{\boldsymbol{\mathcal{W}}}}
\newcommand{\bcX}{\ensuremath{\boldsymbol{\mathcal{X}}}}
\newcommand{\bcY}{\ensuremath{\boldsymbol{\mathcal{Y}}}}
\newcommand{\bcZ}{\ensuremath{\boldsymbol{\mathcal{Z}}}}

\newcommand{\tZ}{\widetilde{Z}}
\newcommand{\tV}{\widetilde{V}}

\newcommand{\Alpha}{\boldsymbol\alpha}
\newcommand{\Nu}{\boldsymbol\nu}

\newcommand{\NPC}{\ensuremath{\mathcal{NP}}-complete}
\newcommand{\NPH}{\ensuremath{\mathcal{NP}}-hard}

%\newcommand{\Sl}{\ensuremath{S_{\ell}}}
%\newcommand{\Slup}{\ensuremath{S_{\ell}^{\margin=0pt,lceil\cdot\rceil}}}
%\newcommand{\Sldown}{\ensuremath{S_{\ell}^{\lfloor\cdot\rfloor}}}
%\newcommand{\Ml}{\ensuremath{M^{(i)}}}
%\newcommand{\MlOdd}{\ensuremath{M^{(i)}_{\Odd}}}
%\newcommand{\MlEven}{\ensuremath{M^{(i)}_{\Even}}}
%\newcommand{\Mt}{\ensuremath{M^{(t)}}}
%\newcommand{\Mi}{\ensuremath{M^{(i)}}}

% Handy definitions
\newcommand{\bigO}[1]{\ensuremath{\mathcal{O} \left( #1 \right)}}
\newcommand{\bigOmega}[1]{\ensuremath{\Omega \left( #1 \right)}}
\newcommand{\bigTheta}[1]{\ensuremath{\Theta \left( #1 \right)}}
\newcommand{\smallO}[1]{\ensuremath{o \left( #1 \right)}}
\newcommand{\smallOmega}[1]{\ensuremath{\omega \left( #1 \right)}}
\newcommand{\abs}[1]{\left| \, #1 \, \right|}
\newcommand{\comb}[2]{\ensuremath{\left( \! \! \begin{array}{c} #1 \\ #2 \end{array} \! \! \right)}}

\renewcommand{\O}{\mathcal{O}}
\newcommand{\Oh}{\mathcal{O}}
\renewcommand{\epsilon}{\ensuremath{\varepsilon}}
\newcommand{\eps}{\ensuremath{\varepsilon}}
\newcommand{\ee}{\ensuremath{\varepsilon}}
\newcommand{\lb}{\log_2}

\newcommand{\dist}{\operatorname{dist}}
\newcommand{\diam}{\operatorname{diam}}
\providecommand{\deg}{\operatorname{deg}}
\newcommand{\out}{\operatorname{out}}
\newcommand{\poly}{\operatorname{poly}}
\newcommand{\polylog}{\operatorname{polylog}}
\newcommand{\polyloglog}{\operatorname{polyloglog}}
\newcommand{\loaddiff}{\mathit{load}\text{-}\mathit{difference}}


\let\oldsqrt\sqrt
\def\hksqrt{\mathpalette\DHLhksqrt}
\def\DHLhksqrt#1#2{\setbox0=\hbox{$#1\oldsqrt{#2\,}$}\dimen0=\ht0
   \advance\dimen0-0.2\ht0
   \setbox2=\hbox{\vrule height\ht0 depth -\dimen0}%
   {\box0\lower0.4pt\box2}}
\renewcommand\sqrt\hksqrt
\renewcommand{\leq}{\leqslant}
\renewcommand{\geq}{\geqslant}
\renewcommand{\le}{\leqslant}
\renewcommand{\ge}{\geqslant}
\renewcommand{\epsilon}{\varepsilon}

% --------------------------------------------------------------------------------
% define \againlem, \againthm, \againpro
\newcommand{\mylem}[2]{\begin{lem}\label{lem:#1}#2\end{lem}}
\newcommand{\againlem}[2]{\noindent\textbf{Restatement of \lemref{#1}}
    (from page \pageref{lem:#1})\textbf{.}\emph{#2}\par}

\newcommand{\mythm}[2]{\begin{thm}\label{thm:#1}#2\end{thm}}
\newcommand{\againthm}[2]{\noindent\textbf{Restatement of \thmref{#1}}
    (from page \pageref{thm:#1})\textbf{.}\emph{#2}\par}

\newcommand{\mycor}[2]{\begin{cor}\label{cor:#1}#2\end{cor}}
\newcommand{\againcor}[2]{\noindent\textbf{Restatement of \corref{#1}}
    (from page \pageref{cor:#1})\textbf{.}\emph{#2}\par}

\newcommand{\mypro}[2]{\begin{pro}\label{pro:#1}#2\end{pro}}
\newcommand{\againpro}[2]{\noindent\textbf{Restatement of \proref{#1}}
    (from page \pageref{pro:#1})\textbf{.}\emph{#2}\par}

\newcommand{\tP}{\ensuremath{\widetilde{\mathbf{P}}}}
\newcommand{\barP}{\ensuremath{\overline{\mathbf{P}}}}
\newcommand{\nE}{\ensuremath{E_{\neq 0}}}


% Changing the footnotes from numbers to symbols
\renewcommand{\thefootnote}{\fnsymbol{footnote}}

% --------------------------------------------------------------------------------
% define \timestamp
\newcount\minute \newcount\hour \newcount\hourMins
\def\now{\minute=\time \hour=\time \divide \hour by 60 \hourMins=\hour \multiply\hourMins by 60
  \advance\minute by -\hourMins \zeroPadTwo{\the\hour}:\zeroPadTwo{\the\minute}}
\def\timestamp{\today\ \now}
\def\today{\the\year-\zeroPadTwo{\the\month}-\zeroPadTwo{\the\day}}
\def\zeroPadTwo#1{\ifnum #1<10 0\fi #1}

% ---------------------------------------------------------------------


%\sloppy

\allowdisplaybreaks[1]   % allow linebreaks even in muli-line equations

%\setstretch{0.95}
\setlength{\marginparwidth}{1.1cm}
